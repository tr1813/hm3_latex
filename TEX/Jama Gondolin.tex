\documentclass[symmetric, a4paper, 11pt]{article}
\title{Jama Gondolin}

\author{Tanguy Racine Imperial College Caving Club}
\usepackage{grffile}
\usepackage{caption}

\usepackage{subcaption}
\captionsetup{compatibility=false}
\usepackage{graphicx}
\usepackage[utf8]{inputenc}
\usepackage{helvet}
\usepackage{multicol}
\usepackage{ragged2e}
\usepackage{cite}
\usepackage{xcolor}
\usepackage{pagecolor}
\usepackage[pages=some,placement=center]{background}
\usepackage{afterpage}
\usepackage{booktabs}
\usepackage{colortbl}
\usepackage{tabularx}
\usepackage{multirow}
\usepackage{wrapfig}
\usepackage[strict]{changepage}



\begin{document}
\maketitle
\paragraph{Location description}
Jama Gondolin is found on the west edge of the plateau. To reach the cave, one follows the hiking trail towards Kuk until reaching a deep E-W valley. Down 100m to the West is V08, a hanging valley with large scree cone and several entrances (JJ1, JJ2 - \cite{hm1}). A roped abseil starts above the steep slope and leads to a rock overhang 10m wide. The high entrance to the cave is located to the top right corner of the rock shelter and reached via pendulum from the abseil route.

\paragraph{Location coordinates} WGS84 - 46$^{\circ}$ 15' 11.24" N 13$^{\circ}$ 45' 26.87" E
\paragraph{Entrance dimensions} 
\begin{itemize}
	\item Height:1.5m high 
	\item Width: 3m wide
\end{itemize}
\paragraph{Surroundings}
V08 is a hanging valley with a steep southern wall, where the entrances to JJ1 and JJ2 are found. Due to the pronounced dip of the Dachstein limestone (~25$^{\circ}$ to the SE), the northern wall is less structurally stable and has eroded bed surfaces, as well as joint surfaces exposed. The scree cone is sourced from this north wall. THere is abundant evidence for corrosion cavities and alcoved extending upwards from the bed surfaces on the cliff side. 

There is also evidence for vadose cave development exposed to the surface due to roof collapse (e.g. JJ1), while relict vadose rift go through a rock pinnacle on the south wall, providing access to the cliff above Jama Planika and Jama Monatip (~100m south).

The entrance to Gondolin is located at the intersection of two joint sets in the limestone. One is NNW-SSE oriented, the other N-S. Geometrically, this provides a buttress of rock similar to Primadona entrance. This systematic location of cave passage or cave entrances (Gondolin, Monatip, Planika, B9 etc... - see Fig topomap) is noteworthy. 

\begin{wrapfigure}{r}{0.5\textwidth}
    
    \frame{\includegraphics[width=\linewidth]{"images/2017/gondolin-2017/gondolin_drawn".pdf}}
    \caption{Survey of Gondolin cave}
    \label{}
\end{wrapfigure}

\paragraph{Entrance series}
A short crawl over sharp cobbles lead to an upwards tube, giving access to a larger space. The first pitch (P7) lands on top of a scree cone with two ways on. Up, leads to the high part of a sinous rift, choked with boulders. The very top of the passage is a relict phreatic tube in the ceiling. 

\paragraph{Rest of cave}
Down, and to the right leads to a continuation of the rift, to the second pitch (P8). This drops to a scree filled alcove and a window onto the third pitch (P13), which is much larger in diameter (6x6m). THe bottom is again a large, steep cone of scree. It is possible to see a parallel, small aven coming in from the N side of the shaft. 
\begin{wrapfigure}{l}{0.5\textwidth}
    
    \frame{\includegraphics[width=\linewidth]{"images/2017/gondolin-2017/gondolin entrance".png}}
    \caption{Entrance of Gondolin cave --- the photo is from a presentation by Maks Merela \cite{maks}}
    \label{}
\end{wrapfigure}
At the bottom an arched passage leads to a low chamber, filled with boulders and cobbles, with some draught. The wind comes in from a sub-human size phreatic tube over to the W side. 

Between the 3rd and 2nd pitches, a small highly sinuous and draughting rift passage leads off the west, with 3 sharp turns before daylight is seen ahead. The draught comes into the cave from this lower entrance.



\paragraph{Cave sediments and formations} 
There are no ice or mineral deposits of note in this cave. The cave floors are exclusively covered in sharp pebble to boulder sized scree. The cave walls show evidence of scalloping in the high parts of the main chamber rift.

\paragraph{Cave geology}
Three stages of cave formation can be identified in this cave. 
\begin{enumerate}
	\item phreatic stage: the highest level or roof of the main sinuous rift is an elliptical ceiling with the typical cross-section of a phreatic conduit. It is located at an elevation of 1772m, this is about 35m above the main phreatic trunk route of Sistem Migovec, which lies at 1735m).
	\item vadose stage: this formed the main desdending rift, with evidence of scalloping, which had water flowing down. The 3rd pitch is very circular, a probably formed through dissolution of the walls due to water spray.
	\item late stage collapse: roof collapse through mechanical failure of the limestone beds is evident in many parts of the cave: boulder cones filling the passage to the top. Cave collapse started during the vadose stage of cave development, taking away the collapse material to deepen the shafts, but continues today. Eventual mechanical failure and response to freeze-thaw due to proximity to the surface continued long after, shaping the scree cones of today. 
\end{enumerate}

We did not witness any cave biology during exploration, but it is very likely to host trogloxene fauna. Its very inaccessible location explains the lack of vegetation infall (ie: the entrance looks down on the Tolminka valley).

\paragraph{Name} The cave was named after the Tolkien lore: \emph{Gondolin} means the hidden valley of stone, and was an elven realm protected by very high mountains. 

\paragraph{Further exploration} The presence of the high phreatic tube as only remaining lead warrants further exploration. This passage leads into the mountain, and lies 140m to the west of over, major caves in Sistem Migovec (M17, NCB). It is also the belief of Tanguy Racine that more, similar cave entrances can be entered on the cliff side between Gondolin and Planika.

\paragraph{Survey} The cave was surveyed by Tanguy Racine, Benjamin Honan and Jack Hare of ICCC. 

\paragraph{Exploration} The cave was explored in two stages: first by Tanguy Racine and Benjamin Honan, the next day Jack Hare joined them both to continue and finish the exploration of the pitch series.

\paragraph{Photography} Cave photography was done by Jack Hare on the second exploratory trip. We photographed the 3rd pitch and final chamber, as well as the entrance. THe photograph that motivated the exploration is also included.

\begin{figure}[t!]

\centering
    \begin{subfigure}[t]{0.52\textwidth}
        \centering
         \includegraphics[width=\linewidth]{"images/2017/gondolin-2017/Jarv_V09 (1)".jpg}
        
        \caption{} \label{V08 valley}
    \end{subfigure}
    \hfill
    \begin{subfigure}[t]{0.46\textwidth}
    \centering
        \includegraphics[width=\linewidth]{"images/2017/gondolin-2017/Gondolin-abseil".jpg}
        \caption{} \label{gondolin abseil}
    \end{subfigure}
    \begin{subfigure}{\textwidth}
    \centering
        \includegraphics[width=\linewidth]{"images/2017/gondolin-2017/Gondolin-pendulum".jpg}
        \caption{} \label{gondolin pendulum}
\end{subfigure}
    \caption{
    \emph{a} The V08 valley with a cave riddled southern wall and massive scree on a foggy day --- Jarvist Frost
    \emph{b} Rigging the abseil down to Gondolin cave from the V08 valley --- Jack Hare
    \emph{c} Ben Honan ascending the 5m pendulum in and out of Gondolin cave --- Jack Hare  }
\end{figure}

\begin{figure}[t!]

\centering
    \begin{subfigure}[t]{0.505\textwidth}
        \centering
         \includegraphics[width=\linewidth]{"images/2017/gondolin-2017/Ben-survey".jpg}
        
        \caption{} \label{surveying gondolin}
    \end{subfigure}
        \hfill
\begin{subfigure}[t]{0.465\textwidth}
\centering
\includegraphics[width=\linewidth]{"images/2017/gondolin-2017/Ben_Gondolin".jpg} 
 \caption{}\label{the third pitch}
\end{subfigure}
    \vspace{0cm}
    \begin{subfigure}[t]{\textwidth}
    \centering
       
        \includegraphics[width=\linewidth]{"images/2017/gondolin-2017/Tanguy-Gondolin".jpg}
        \caption{} \label{bottom of gondolin}
    \end{subfigure}
    \caption{
    \emph{a} Ben Honan surveying at the permanent survey station (bottom of Gondolin)
     \emph{b} The third pitch in Gondolin, with several side avens and cupolas lands on a large rock pile.
     \emph{c} Although promising looking, the bottom of Gondolin was choked with frost shattered rock piles --- Jack Hare}
\end{figure}
\clearpage
\bibliographystyle{apalike}
    \bibliography{biblio_hm3}
\end{document}